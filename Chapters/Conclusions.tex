% Chapter Template

\chapter{Conclusioni} % Main chapter title

\label{Conclusions} % Change X to a consecutive number; for referencing this chapter elsewhere, use \ref{ChapterX}

 %----------------------------------------------------
 
 
Con il progredire delle tecnologie digitali e di manifattura disponibili risulta utile conoscere come queste possano essere integrate nella pratica odonto\-iatrica, per identificare le fasi in cui ne è possibile l'integrazione e valutare se effettivamente possa derivarne un vantaggio pratico per il clinico ed il paziente.\\
Le possibili applicazioni della manifattura additiva in odonto\-iatria sono molte\-plici, dato l'ampio numero di procedure in cui l'odontoiatra si trova impegnato a progettare e realizzare dei prodotti personalizzati per il paziente. I vantaggi pratici delle tecnologie digitali sono in primis la riduzione dei materiali e dello spazio necessari alla presa dell'impronta, che essendo digitale può essere agevolmente conservata per un tempo indefinito in un archivio, ed all'occorrenza stampata, anche a distanza di anni senza alcun deterioramento.\\
La stessa impronta può essere agevolmente scambiata digitalmente con l'odonto\-tecnico o con i colleghi, facilitando lo scambio di informazioni e riducendo la possibilità di errori da parte dell'operatore.\\
Le possibilità di personalizzazione del trattamento sono però la caratteristica più interessante dell'approccio digitale all'odontoiatria. La possibilità di realizzare in studio brackets ortodontici personalizzati, che possono essere ulteriormente modificati per gestire ogni step del trattamento, è una prospettiva sicuramente interessante che merita un approfondimento.\\ La realizzazione di impianti postestrattivi anatomici in studio è stata già testata in almeno due trial clinici \parencite{Reference85}, \parencite{Reference87} e, anche se eseguita con tecniche diverse, dimostra come l'alternativa personalizzata alla classica fixture sia una possibilità reale. Questo approccio ha mostrato potenzialità di poter essere una valida risorsa in alcune situazioni riabilitative, perciò ulteriori studi sono necessari per validarne l'efficacia e fornire delle linee guida operativa per l'uso clinico.\\
L'odontoiatria protesica già da tempo si avvale dei tecnologie CAD-CAM per il supporto durante alcune fasi del trattamento. Da questa prospettiva le tecnologie di manifattura additiva vanno valutate per ricercare effettivi vantaggi nei confronti della manifattura sottrattiva. Il risparmio di materiale è sicuramente un vantaggio della manifattura additiva, così come la capacità di poter realizzare forme estremamente complesse. La precisione è un elemento fondamentale in protesi, e dagli studi analizzati si evince che in molte circostanze la precisione e l'accuratezza raggiunte dalla manifattura additiva sono al livello della manifattura sottrattiva o poco più accurate, in un range generalmente ritenuto adatto all'utilizzo clinico. \\
La possibilità di effettuare una scansione intraorale e compararla digitalmente con le precedenti è interessante, e può risultare estremamente utile nel monitoraggio della crescita del piccolo paziente, specie in ortodonzia.\\
Con l'odontoiatra che spesso si trova già nelle condizioni di utilizzare queste tecnologie è importante promuovere la competenza nella gestione dei dati digitali. Di primaria importanza è la garanzia della privacy del paziente, data la facilità con cui il dato digitale può essere scambiato. La corretta gestione delle immagini mediche e dei modelli è importante per preservare l'informazione originale il più possibile inalterata. Il processo di stampa infine deve essere accurato, con grande riguardo alla condizione di operatività della stampante ed ai parametri di stampa. La letteratura fornisce conoscenze generali e spunti utili, ma questi vanno sempre integrati con le istruzioni del produttore e con le specifiche della macchina.\\
Le prospettive future includono la stampa di scaffold per la rigenerazione dei tessuti e, ancora oltre, la produzione in vitro di tessuti ed organi pronti al trapianto sul paziente. Entrambe questi filoni di ricerca si avvalgono di tecniche di manifattura additiva, e c'è già chi prospetta che in futuro la stampa di tessuti possa essere integrata nella routine clinica \parencite{Reference142}.\\
Queste sono prospettive sicuramente stimolanti e che meritano una discussione più attenta, per le varie implicazioni che queste tecnologie potranno avere in futuro. La tecnologia progredisce velocemente e con lei i macchinari ed i software correlati, per cui le metodiche qui trattate rappresentano solo uno dei tanti approcci che possono essere utilizzati dal clinico. Al centro del trattamento resta sempre il paziente, perciò ogni scelta terapeutica va attentamente valutata e messa in atto soltanto se questa apporta un reale miglioramento della qualità di vita del paziente.
