% Chapter Template

\chapter{Free (Libre) Software} % Main chapter title

\label{Capitolo1} % Change X to a consecutive number; for referencing this chapter elsewhere, use \ref{ChapterX}
 
 %----------------------------------------------------
In the current software landscape, there are various licenses that regulate software's intellectual property and describe the conditions of use and redistribution. Generally we distinguish software with commercial licenses and software with free licenses. \\
The idea of free software comes from \emph{Richard Stallman}, MIT researcher, in 1983, with the aim of creating, distributing and maintaining free software that can be used by all users \parencite{Reference57}, as opposed to software under commercial license that began to be born in those years.\\
Stallman has subsequently founded the \emph{Free Software Foundation}, which is a non-profit organization committed to pursuing the idea and principles of free software \parencite{Reference58}. The FSF also draws free licenses for the protection of copyright; free software is in fact a licensed software, which sets rules on its modification and distribution that ensure that the software continues to be free. \\ There are several free licenses produced by the FSF \parencite{Reference59} and by other entities \parencite{Reference60}, the most common being the GPLv3 \parencite{Reference61}.\\
The use of GPL licenses guarantees that the software released will always be a free software, regardless of changes and redistribution. This is called \emph{Copyleft} and guarantees the freedom of users, opposing to the Copyright.\\
A free license is defined as such if it complies with the \textbf{four freedoms} described by the FSF: \\
\begin{itemize}
\item freedom to use the software for any desired purpose, by any user, on any desired device;
\item freedom to access the source code and to study and modify the software according to your needs;
\item freedom to distribute to the software with whom you want;
\item freedom to distribute the changes made to the software in the way you want, even without Copyleft, but without the obligation to release in non-free form.
\end{itemize}
Software that meets these requirements is called free software. \\
Free software is different from \emph{open-source} software, as in free software there is a strong reference to ethics and freedom of use, while open-source licenses focus on accessibility to source code, which is nevertheless a requirement of free software.

\section{Why to use free software?}
The use of free software is encouraged wherever possible in this manuscript, for the guaranteed freedoms and because it is considered appropriate to make this work widely available and reusable. Finding software at no cost facilitates, in our case, the exploration of alternative routes for the use of diagnostic images, while the \emph{Rep-Rap} project \parencite{Reference62} allow free use of the information regarding additive manufacturing instruments. \\
The freedom to adapt existing hardware and software relieves the users from recreating basic functions, which are already tested in the existing tools, and allows them to focus on solving the specific problem.
In our case many open source software does not contain specific functions for dentistry, and some neither for the medical field in general. With the available source code however, it is possible to make changes or write new functions that allow a fast progression from idea to experimentation (for example \parencite{Reference63}, \parencite{Reference64}). This type of freedom is greatly reduced in commercial software, in which the integration of new functions is possible only by the software manufacturer. \\
Open source software is also often supported by communities of users who contribute to the quality of software, and anyone can check the source code to validate its security, and in the case of flaws, these can be often quickly fixed.\\
Free and open manuscripts are part of general knowledge, to help any user who needs them. They must therefore be preserved and expanded, to ensure that knowledge is not lost, and instead it is made more accessible to accelerate the general process of improvement to which the human beings boldly aims.