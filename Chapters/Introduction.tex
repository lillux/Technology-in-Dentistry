% Chapter Template

\chapter{Introduction} % Main chapter title

\label{Introduzione} % Change X to a consecutive number; for referencing this chapter elsewhere, use \ref{ChapterX}
 
 %----------------------------------------------------
 
 
This work was made to give an insight into the use of \emph{medical modeling} and of its dental applications in combination with additive manufacturing technologies, which have potential for significant impact on clinical practice and dental research.\\
The digital management of medical diagnostic images is treated focusing on the practical use, through the use of segmentation software and extraction of three-dimensional digital models of parts of the human organism. There is an overview on the management of digital models to obtain, through the 3D printing process, real three-dimensional models of anatomical parts and pathological lesions.\\
Diagnostic images processing is a process that generates a large amount of data, which can be further processed to improve patient management and treatment. The three-dimensional visualization of the patient's reconstructions gives the physician an in-depth view of the clinical situation, and makes it possible to design accurate and often minimally invasive treatment plans. The clinical researcher can now use various technologies to analyze images and produce physical object, which serve as an aid to diagnosis and therapy.\\
Concepts of \emph{computer vision} and \emph{computer graphics} make possible the visualization of images and 3D reconstructions, as well as the production and advanced elaboration of the models. Modern additive manufacturing techniques bring the virtual model into reality, using various materials to generate real models from digital counterparts. There is therefore the possibility to touch the patient's reconstructions, for a better study of the case, for a better explanation of the treatment plan to the patient and for the planning of surgical interventions.\\
The additive manufacturing in the medical field goes beyond the printing of models made of thermoplastic material. Biofabrication is a recent discipline which, using specific tools for additive manufacturing and knowledge gained in the fields of regenerative medicine and tissue engineering, aims to reproduce in vitro tissues and organs of the human organism, which can be implanted on patients instead of prostheses or organs taken from donors. \\
A good diagnosis starts with the analysis of patient data. With the growth of data available on large numbers of patients, at the populations level, it is difficult for the human being to find every correlation between them. For this reason, algorithms have been developed to correlate and classify data. The use of this amount of information can have serious implications in research, among all in the pharmacological research and in the research of molecular markers of disease and tumors. \\
This amalgamation of technologies gives to the doctor, today more than ever, the opportunity to provide the patient with a treatment of excellence, from the diagnosis to the follow-up. Although the learning curve for the use of some of the software and technologies used here is steep, the possibilities that they provide for personalizing patient treatment are relevant. \\
The aim of this work is also to make the workflow as consistent as possible with the ideals of the \emph{Free Software Foundation} (FSF), to facilitate the adoption of the procedures described here and to give a small contribution to the diffusion of knowledge and awareness of a field with strong potential for the protection of human health.
