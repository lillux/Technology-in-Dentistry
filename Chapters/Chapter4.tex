% Chapter Template

\chapter{3D Models from Medical Imaging} % Main chapter title

\label{Chapter4} % Change X to a consecutive number; for referencing this chapter elsewhere, use \ref{ChapterX}
 
 %----------------------------------------------------


Having understood the principles of medical diagnostics and digital modeling, we continue with a use case, which shows how to derive a 3D model from a series of diagnostic images. The procedure described here is used to give an overview of the main functionalities of the software and the basic techniques of image and model management with the aim of providing a generic workflow and some insights on possible variations of this for the optimal resolution of cases specific. \\
The case presented here shows how to get a model of a jaw and how to prepare it for printing. The presented procedure is well suited to the extraction of tissue models of organs with a strong contrast to surrounding ones, such as bone tissue in CT.

\section{Image anonymization}
Before starting to work with images we must ensure that these do not contain sensitive data with which we can trace the patient's identity. In this case we will use the open-source software \emph{DICOM Confidential}, developed by the team \emph{Data Intensive Research of the University of Edinburgh} \parencite{Reference46} \parencite{Reference146}. The software allows you to load a folder containing the images, which are then processed according to the directives entered. \\
When the software is opened, the graphical user interface (GUI) appears.
Here you can upload the folder containing the set to be anonymized. \\
You can set the entries \emph{Policy URI} and \emph{Workflow file}. These are the indications on the workflow to be performed and on the anonymisation operations to be performed on the images. The installation software provides standard templates to be used. These can be modified as required, and are located in the path \path | C: Userpath \ DICOM Confidential | (where \path | C: Userpath \ DICOM Confidential \ | is the folder where the software was installed). \\
The output path is \path | C: Userpath \ DICOM Confidential \ data \ ANONYMISED |. Using the standard configuration files, each anonymized set will be recognized by the software for the opening of the DICOM as belonging to the single patient "ANONYMISED", to which will be added all the anonymized studies. To be taken into account during the organization of the system.
\begin{figure}[h]
	\centering
	\includegraphics[width=0.9\textwidth, keepaspectratio]{gui_dicomConf}
    \caption{Schermata principale del software \emph{DICOM Confidential}}
    \label{fig:gui_dicomConf}
\end{figure}

\section{Anonimized Images}
Is possible to download anonymised datasets from the \emph{Cancer Imaging Archive} \parencite{Reference47}, a database maintained by the \emph{The National Cancer Institute} (NCI) and the \emph{University of Arkansas for Medical Sciences} (UAMS) to support multidisciplinary research \parencite {Reference48}, together with the \emph{Cancer Genome Atlas} dataset \parencite{Reference49}. \\
There are several studies available with the various imaging techniques available, which give the opportunity to deepen the analysis of various segments of the body and oncological diseases that may occur.
To use the images, follow the simple procedures described on the website. Once the data is obtained, it can be loaded directly onto the visualization software.
                             
\section{Opening images}
The images obtained can then be uploaded to 3D Slicer for viewing and processing. We will use a set of images downloaded from the \emph{Cancer Imaging Archive} portal, located within the \path | TCGA - HNSC collection, Subject ID: TGCA-BA-6868 |, scan \emph{Neck BW Axial}. \\
The software uses the \emph{DICOM} \parencite{Reference50} module to load and manage diagnostic image sets. The \emph{Volume Rendering} module can be used to display a rendered image. \\
With the \emph{Crop Volume} module it is possible to cut the volume part of our interest (Region of Interest, ROI), to lighten the computer from data that are not needed at the moment, which will be very useful later when we work with the models. In this example we want to create a model of \emph{\textbf{mandibular bone}}, so we orientate the ROI so that it contains the maxillary bones and we apply the modification.
\begin{figure}[h]
\centering
\includegraphics[width=0.8\textwidth, keepaspectratio]{crop1}
\caption{crop effettuato sulle immagini per selezionare le arcate mascellari}
\label{fig:crop}
\end{figure}
\vspace{-10pt}
\section{Images Segmentation and Model Creation}
We can then start segmentation, using the \emph{Segment Editor} module. This module contains several tools that allow us to highlight the areas of the images that interest us. Since we want to extract the model of a bone, the jaw, a quick method is to use the \emph{bone density range} to quickly highlight the region of our interest. Then with the \emph{Add} button we add a segment and select the \emph{Threshold} tool which allows us to select a density range to highlight. The \emph{Data Probe} tool at the bottom left gives us information about the point in the image where the mouse point is located, and contains an entry indicating the density at the point.
In selecting the threshold we look for a compromise between completeness in capturing details and cleaning the image. We have the possibility to perform a manual finishing of the segmentation, so we can allow ourselves to leave some unselected area from the threshold to have a greater cleaning between the parts to be segmented. We consider that the selected areas, to be treated as single areas must be separated, except if a segment is used for each area; in that case the selection voxels can be adjacent without merging the volumes into a single object.
However, it is important to select the areas of interest in the best way, and in the CT we are segmenting a delicate point for the correct jaw separation in the area of the condyles, where the selection is continuous with that of the anterior wall of the glenoid bone cavity thunderstorm. Another point that could be separated is the posterior region of the dental arch, at the level of the occlusal plane, where the teeth of the jaw could be joined to the antagonists.
\begin{figure}[h]
\centering
\includegraphics [width=0.7\textwidth, keepaspectratio]{origin_label}
\caption{Selection of mandibular bone and lower dental elements; \emph{threshold} 220 - 3071 HU; removal of small voxel groups: \emph{Island}> \emph{remove small island (minimum voxel=1000)}.}
\label{fig: origin_label}
\end{figure}
\\
Once the threshold has been applied, we see that a mask has been created on voxels that fall within the selected density range. This mask must be modified manually, adding any missing areas and separating the parts of our interest from the adjacent ones. By clicking on the button \ emph {Show 3D} we can see the preview of the model created by the segmentation; this option is useful to see if the selected areas correspond to the model we want to obtain, but when modifying the mask it is better to deactivate the option to save resources, and activate it only when necessary. \\
The removal of the contact areas between two parts, in our case the mandibular condyle and the condyle of the temporal vein, performed in 3D Slicer to speed up the subsequent processing of the model. A model could have been created already after the use of the threshold, but the separation of two models in software such as Blender is longer and more complicated than the simple selection/deselection of voxels that can be performed in 3D Slicer.

\begin{figure}[h]
\centering
\includegraphics[width=0.4\textwidth, keepaspectratio]{fuso_condi}
\caption{3D reconstruction of the threshold segmentation of the mandible. The mandible is fused to the temporal bone.}
\label{fig:fuso_condi}
\end{figure}
\vspace{-5pt}
We therefore aim to obtain from Slicer a 3D model as precise and clean as possible, to reduce the duration of subsequent steps, but also to create good quality segmentation datasets, which are useful for other purposes (data analysis, training set for Neural Network). \\
The \emph{Erase} tool allows you to remove the voxels that you do not need to select from the mask. \\
The \emph{Paint} tool lets you color voxels of interest that are not selected layers from the threshold. \\
After finishing, we can use the \emph{Island} tool with the \emph{Keep Selected Island} option enabled; clicking on the jaw selection mask. If this is sepatated from the rest of the skull we will have correctly performed the separation, which we can control by displaying the model with the button \emph{Show 3D}. If we also want the model of the rest of the skull, we can click on the \emph{Undo} button to go back one step and retrieve the island containing the jaw and part of the skull.\\
We can create a segment from the mask portion of the skull. We add a segment from the \emph{Add} button and select it; using the \emph{Island} tool with the \emph{Add Selected Island} option we click on the skull mask and it will be added to the new segment. Separating objects into segments causes the spatial relationship to be lost between the parts, so if the parts need to be in a particular position, the \emph{Merge into single file} option must be selected to export to a single file. It is then possible to separate this unique model in its two components with the Blender software. \\
Then go to the module \emph{Segmentation}, where from the menu on the left we find the window \emph{Export to File}; select the destination folder and export the files in .stl format.
\vspace{-10pt}
\begin{figure}[h!]
\centering
\includegraphics [width=0.4\textwidth, keepaspectratio]{sepa_condi}
\caption{Mandible separated from the temporal bone. The separation was done manually with the erase and paint tools. After separation a Segment for the jaw (yellow) and one for the skull (blue) were created.}
\label {fig: sepa_condi}
\end{figure}
\vspace{-15pt}
\newpage
\section{Model processing}
We import the model obtained in to MeshMixer and in the \emph{Analysis} menu we select the \emph{Inspector} tool. This function essentially has the task of detecting faults that must be resolved in order to have a closed model (\emph{manifold}) and to remove components separated from the main mesh. It is useful for preparing a model for 3D printing, where it is necessary that the model is precisely closed,  \emph{watertight}. This tool is useful for solving small problems with the mesh, but complex situations may require manual repair, which can be performed in this software as well as with Blender and MeshLab.

\begin{figure}[h]
\centering
\includegraphics[width=0.8\textwidth, keepaspectratio]{inspector}
\caption{Strumento \emph{Inspector} in MeshMixer. Ogni indicatore viola può essere cliccato per riparare un difetto della mesh.}
\label{fig:inspector}
\end{figure}

The model can then be exported to Blender to perform a smoothing. In the \emph{Modifiers} menu, select the \emph{Smooth} modifier. This is a function that smoothes the surface of the object, causing a slight decrease in the volume. In the instrument menu we can set the number of filter passages on the model; we find a compromise between sanding and maintaining dimensions. Another way to smooth the surface is to do it manually, with the sculpting tools \emph{sculpting} present in both Blender and MeshMixer. \\
After having polished the model, we inspect it to evaluate its quality; you can use the MeshLab software to compare two meshes. In our case we will measure the difference between the original model and the model obtained after 50 iterations of the \emph{smooth} modifier in Blender. When the models are aligned with each other, the procedure is to use the Hausdorff Distance filter to evaluate the distance between the two meshes \parencite{Reference90}, \parencite{Reference91}. The software will return measurements related to the mesh sampling. \\
\begin{figure}[h]
\centering
\includegraphics[width=0.8\textwidth, keepaspectratio]{hausdorff_smooth_clean}
\caption[LoF entry]{Valutazione della distanza di Hausdorff tra la mesh originale e la stessa mesh dopo 50 step di "Smooth" in Blender. La scala va da rosso (errore 0) a blu (errore massimo)}

\begin{lstlisting}
Hausdorff Distance computed
Sampled 4657134 pts (rng: 0) on smooth_clean_Segmentation_bone.stl
searched closest on clean_Segmentation_bone.stl
min: 0.000000; max: 0.695190; mean: 0.116956; RMS: 0.154653

Values w.r.t. BBox Diag (181.472809)
min: 0.000000; max: 0.003831; mean: 0.000644; RMS: 0.000852 
Applied filter Hausdorff Distance in 29689 msec

Quality Range: 0.000000 0.666229; Used (0.006929 0.335713)
percentile (5.000000 95.000000) 
Applied filter Colorize by vertex Quality in 51 msec
\end{lstlisting}
\label{fig:hausdorff_smooth_clean}
\end{figure}
The mesh can then be colored to visually evaluate the discrepancy with the original mesh. To do this from the \emph{Filter} menu, select \emph{Color Creation and Processing} -> \emph{Colorize by Vertex Quality}. We can render the histogram from the menu \emph{Render} -> \emph{Show Quality Histogram}. \\
Keep in mind that the color scale goes from red to blue, where the \emph{red} indicates maximum correspondence with the original mesh, while the \emph{blue} indicates a greater distance from the original mesh. The value is shown in units of the model, in this case millimeters. \\
As an alternative to MeshMixer it is possible to use MeshLab to perform more advanced mesh operations.
To remove regions separated from the main mesh, use the \emph{Filter} -> \emph{Cleaning and Reapiring} -> \emph{Remove Isolated Pieces} tool, setting the minimum number of faces at a fairly high level, based on the number of faces shown in the bottom tool-bar. \\
To obtain a \emph{mesh 2-manifold} \parencite{Reference92}, \parencite{Reference93}, in practical terms a \emph{mesh closed}: from the menu \emph{Filter} -> \emph{Cleaning and Repairing } use the functions: \emph{Remove Duplicate Faces}, \emph{Remove Duplicate Vertex}, \emph{Remove Unreferenced Vertices}, \emph{Remove Faces From Non Manifold Mesh}, \emph{Remove t-vertices From Non Manifold Edges}. \\ These commands clean up the mesh, and to rebuild the 2-manifold mesh we use the command \emph{Filter} -> \emph{Remeshing, Simplification and Reconstruction} -> \emph{Screened Poisson Surface Sampling} \parencite{Reference95}, \parencite{Reference96}. This clean, 2-manifold reconstruction can be used for further processing. By performing an assessment of the adequacy of the reconstruction, it is noted that the discrepancy with the original is very low.
\begin{figure}[h]
\centering
\includegraphics[width=0.8\textwidth, keepaspectratio]{noSeparate_poisson_clean_hausdorff}
\caption[LoF entry]{Calcolo della distanza di Hausdorff dopo l'applicazione dell'algoritmo \emph{Screened Poisson Surface Sampling}}

\begin{lstlisting}
Hausdorff Distance computed
Sampled 4195438 pts (rng: 0) on Poisson mesh searched 
closest on Segmentation_bone.stl
min : 0.000000 max 0.432402 mean : 0.019149 RMS : 0.031849

Values w.r.t. BBox Diag (182.226028)
min : 0.000000 max 0.002373 mean : 0.000105 RMS : 0.000175 
Applied filter Hausdorff Distance in 20254 msec
\end{lstlisting}
\label{fig:noSeparate_poisson_clean_hausdorff}
\end{figure}

We can make additional checks with the MeshMixer \emph{Inspector} tool to assess the need for further mesh refinements.
The model must then be exported in .stl format to prepare it for 3D printing.
\newpage

\section{Slicing}
The digital model of the Jaw was obtained from the TC and finished to ensure a model suitable for printing. We will be slicing the model using the Cura software.
Load the model and select the \emph{Print Setup} entry \emph{Custom}, to have the ability to fine-tune the print parameters, which are initially hidden and must be activated. \\
The settings to be adjusted depend on several variables, including:

\begin{itemize}
\item the characteristics of the printer;
\item the characteristics of the model to be printed;
\item the print material;
\item the accuracy and the mechanical properties expected from the object that we approach to print.
\end{itemize}

The environmental conditions in which you work are also to be taken into account, because for example some parameters may change between a print executed in a closed printer or in an open one.
The calibration of the printer and the knowledge of the latter's specifications are fundamental for a good printing result. Care will generate printing instructions for the printer based on the printer specifications we entered in the \emph{Printers} menu > \emph{Machine Settings}. These settings are often updated with the addition of the most popular printer data, but for self-assembled printers the parameters must be entered manually. \\
With the calibrated printer and the information correctly entered in Cura, the model will be displayed in the software and will be sliced with the parameters entered. Once the slicing procedure has been completed we will be able to see the model layer by layer (\emph{Layer View}).
The so obtained \emph{\textbf{G-code}} can be used by the printer to produce the model.
\begin{figure}[t]
\centering
\includegraphics[width=\textwidth, keepaspectratio]{slicing}
\caption{Slicing della mandibola in Cura.}
\label{fig:slicing}
\end{figure}

